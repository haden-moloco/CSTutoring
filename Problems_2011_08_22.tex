\documentclass[12pt]{article}
%\usepackage{fullpage}
\usepackage{epic}
\usepackage{eepic}
\usepackage{amsmath,amsthm,amsfonts,amssymb}
\usepackage{mathrsfs}
\usepackage{tweaklist}
\usepackage{algorithmic,algorithm}
\usepackage{ulem}

%%%%%%%%%%%%%%%%%%%%%%%%%%%%%%%%%%%%%%%%%%%%%%%%%%%%%%%%%%%%%%%%
% This is FULLPAGE.STY by H.Partl, Version 2 as of 15 Dec 1988.
% Document Style Option to fill the paper just like Plain TeX.

\typeout{Style Option FULLPAGE Version 2 as of 15 Dec 1988}

\topmargin 0pt
\advance \topmargin by -\headheight
\advance \topmargin by -\headsep

\textheight 8.9in

\oddsidemargin 0pt
\evensidemargin \oddsidemargin
\marginparwidth 0.5in

\textwidth 6.5in
%%%%%%%%%%%%%%%%%%%%%%%%%%%%%%%%%%%%%%%%%%%%%%%%%%%%%%%%%%%%%%%%

\pagestyle{empty}
\setlength{\oddsidemargin}{0in}
\setlength{\topmargin}{-0.8in}
\setlength{\textwidth}{6.8in}
\setlength{\textheight}{9.5in}
\setlength{\parindent}{0in}
\setlength{\parskip}{1ex}

% -*- Tex -*- macros for no-fill, built on Plain

\newenvironment{code}{\samepage \vskip 0.1in \tt\nofill}{\endnofill \vskip 0.1in}
\newenvironment{tightcode}{\samepage \tt\nofill}{\endnofill}

\catcode`\@=11\relax	     % allow @ in macro names

% a simple no-fill, used as
%
% \nofill
% Here is some no-fill text.
%    This line starts with three spaces.
% This is the    last line.  It has embedded spaces which will appear.
% \endnofill
%
% You might want to switch to a fixed-width font when you use
% these macros.

\def\nofill {%
  \begingroup
    % spaces made active so they can be tested for at start of lines
    %  and so multiple spaces aren't collapsed to one.
    \obeyspaces
    % nofill via every line a par; if spaces start next line
    %   we do a \noindent to make sure a par gets started.
    \parskip=\z@
    \parindent=\z@
    \let\p@r=\par
    \def\par{\p@r \ifspacenext{\noindent}{}}%
    \obeylines}

\def\endnofill{
  \endgroup}

% \ifspacenext is used as:
%     \ifspacenext {<true tokens>}{<false tokens>}
% It executes the true tokens if
% the next character is an active space, the false tokens
% otherwise.  The same technique can be used to check for any
% character, not just active space.

% You can NOT nest uses of \ifspacenext!

% get a token which is \ifx-equal to active space, so we can test
% for it
{\obeyspaces\global\let\sp@ce= \relax}

\def\ifspacenext #1#2{%
  \def\truet@ks{#1}%
  \def\falset@ks{#2}%
  \futurelet\next\ifsp@cenext}
\def\ifsp@cenext {%
  \ifx\next\sp@ce \truet@ks \else \falset@ks \fi}

\catcode`\@=12\relax	% make @ inaccessible again


\newtheorem{prop}{Proposition}
\newtheorem{lemma}{Lemma}
\newtheorem{thm}{Theorem}

\newcommand{\cost}{\operatorname{cost}}
\newcommand{\opt}{\mathrm{OPT}}

\usepackage{graphicx}
\begin{document}

\setlength{\fboxrule}{.5mm}\setlength{\fboxsep}{1.2mm}
\newlength{\boxlength}\setlength{\boxlength}{\textwidth}
\addtolength{\boxlength}{-4mm}

\def\ind{\hspace*{0.3in}}
\def\gap{0.2in}



\section{Finding Prime Numbers}

\subsection*{Description}
You want to find $n$-th prime number, given a positive integer $n$. 
For your information, the first ten prime numbers are:
$2, 3, 5, 7, 11, 13, 17, 19, 23, 29$. 

\subsection*{Input Format}
A single integer $n$ is given. You may assume that $1 \leq n \leq 1,000$. 

\subsection*{Output Format}
Output a single integer, which is the $n$-th prime number.

\subsection*{Sample Input 1}
\begin{verbatim}
1
\end{verbatim}

\subsection*{Sample Output 1}
\begin{verbatim}
2
\end{verbatim}

\subsection*{Sample Input 2}
\begin{verbatim}
5
\end{verbatim}

\subsection*{Sample Output 2}
\begin{verbatim}
11
\end{verbatim}



\newpage


\section{Number Tornado}

\subsection*{Description}
Given an odd number $n$, you want to output all natural numbers from $1$ to $n^2$, inclusive, in a particular pattern, an $n$ by $n$ grid.
 
You want to have $1$ at the very middle of the grid, and then you are going to have the following numbers in a tornado-like pattern, clockwise.  See the sample output for more information.

\subsection*{Input Format}
You are given a single integer $n$, where $3 \leq n \leq 13$, and $n$ is always odd. 
\subsection*{Output Format}
You output the numbers as if they were put in $n$ by $n$ grid.

\subsection*{Note}
As in sample outputs, you may need to add an extra whitespace in front of smaller numbers as their lengths are shorter. 
You can specify how long each decimal number should be in characters, including the white-spaces, by using \%\#d where \# is the length you specify.
For instance, if you used \%3d and output value 3, then it will first print two white-spaces and then 3, to make its total length 3 bytes long.


\subsection*{Sample Input 1}
\begin{verbatim}
3
\end{verbatim}

\subsection*{Sample Output 1}
\begin{verbatim}
9 2 3
8 1 4
7 6 5
\end{verbatim}

\subsection*{Sample Input 2}
\begin{verbatim}
5
\end{verbatim}

\subsection*{Sample Output 2}
\begin{verbatim}
25 10 11 12 13
24  9  2  3 14
23  8  1  4 15
22  7  6  5 16
21 20 19 18 17
\end{verbatim}



\newpage



\section{Pascal's Triangle}

\subsection*{Description}
Given an integer $n$, output the first $n$ lines of Pascal's triangle.

\subsection*{Input Format}
A single integer $n$ is given. You may assume that $1 \leq n \leq 20$. 

\subsection*{Output Format}
Output Pascal's triangle.

\subsection*{Sample Input}
\begin{verbatim}
5
\end{verbatim}

\subsection*{Sample Output}
\begin{verbatim}
1
1 1
1 2 1
1 3 3 1
1 4 6 4 1
\end{verbatim}






\end{document}

