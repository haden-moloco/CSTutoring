\documentclass[12pt]{article}
%\usepackage{fullpage}
\usepackage{epic}
\usepackage{eepic}
\usepackage{amsmath,amsthm,amsfonts,amssymb}
\usepackage{mathrsfs}
\usepackage{tweaklist}
\usepackage{algorithmic,algorithm}
\usepackage{ulem}

%%%%%%%%%%%%%%%%%%%%%%%%%%%%%%%%%%%%%%%%%%%%%%%%%%%%%%%%%%%%%%%%
% This is FULLPAGE.STY by H.Partl, Version 2 as of 15 Dec 1988.
% Document Style Option to fill the paper just like Plain TeX.

\typeout{Style Option FULLPAGE Version 2 as of 15 Dec 1988}

\topmargin 0pt
\advance \topmargin by -\headheight
\advance \topmargin by -\headsep

\textheight 8.9in

\oddsidemargin 0pt
\evensidemargin \oddsidemargin
\marginparwidth 0.5in

\textwidth 6.5in
%%%%%%%%%%%%%%%%%%%%%%%%%%%%%%%%%%%%%%%%%%%%%%%%%%%%%%%%%%%%%%%%

\pagestyle{empty}
\setlength{\oddsidemargin}{0in}
\setlength{\topmargin}{-0.8in}
\setlength{\textwidth}{6.8in}
\setlength{\textheight}{9.5in}
\setlength{\parindent}{0in}
\setlength{\parskip}{1ex}

% -*- Tex -*- macros for no-fill, built on Plain

\newenvironment{code}{\samepage \vskip 0.1in \tt\nofill}{\endnofill \vskip 0.1in}
\newenvironment{tightcode}{\samepage \tt\nofill}{\endnofill}

\catcode`\@=11\relax	     % allow @ in macro names

% a simple no-fill, used as
%
% \nofill
% Here is some no-fill text.
%    This line starts with three spaces.
% This is the    last line.  It has embedded spaces which will appear.
% \endnofill
%
% You might want to switch to a fixed-width font when you use
% these macros.

\def\nofill {%
  \begingroup
    % spaces made active so they can be tested for at start of lines
    %  and so multiple spaces aren't collapsed to one.
    \obeyspaces
    % nofill via every line a par; if spaces start next line
    %   we do a \noindent to make sure a par gets started.
    \parskip=\z@
    \parindent=\z@
    \let\p@r=\par
    \def\par{\p@r \ifspacenext{\noindent}{}}%
    \obeylines}

\def\endnofill{
  \endgroup}

% \ifspacenext is used as:
%     \ifspacenext {<true tokens>}{<false tokens>}
% It executes the true tokens if
% the next character is an active space, the false tokens
% otherwise.  The same technique can be used to check for any
% character, not just active space.

% You can NOT nest uses of \ifspacenext!

% get a token which is \ifx-equal to active space, so we can test
% for it
{\obeyspaces\global\let\sp@ce= \relax}

\def\ifspacenext #1#2{%
  \def\truet@ks{#1}%
  \def\falset@ks{#2}%
  \futurelet\next\ifsp@cenext}
\def\ifsp@cenext {%
  \ifx\next\sp@ce \truet@ks \else \falset@ks \fi}

\catcode`\@=12\relax	% make @ inaccessible again


\newtheorem{prop}{Proposition}
\newtheorem{lemma}{Lemma}
\newtheorem{thm}{Theorem}

\newcommand{\cost}{\operatorname{cost}}
\newcommand{\opt}{\mathrm{OPT}}

\usepackage{graphicx}
\begin{document}

\setlength{\fboxrule}{.5mm}\setlength{\fboxsep}{1.2mm}
\newlength{\boxlength}\setlength{\boxlength}{\textwidth}
\addtolength{\boxlength}{-4mm}

\def\ind{\hspace*{0.3in}}
\def\gap{0.2in}



\begin{center}\framebox{\parbox{\boxlength}{\bf
CS Tutoring \hfill Author: Haden Lee\\ 
Science Fair \hfill Date: 10/03/2011}}\end{center}
\vspace{5mm}


\section{Partial DNA}

\subsection*{Description}
Given a long sequence of DNA as a string, output the first 10 chars followed by three dots followed by the last 10 chars. 

\subsection*{Note}
We are using a file input (use FILE, fopen, fscanf) and the standard console output (printf).

\subsection*{Input Format}
You are given a string whose length is at least 20 bytes long, consisting only of A, C, G, and T.

\subsection*{Output Format}
Output as described.

\subsection*{Sample Input (input.txt)}
\begin{verbatim}
ACACAGGGTTAAAAATTTCG
\end{verbatim}

\subsection*{Sample Output}
\begin{verbatim}
ACACAGGGTT...AAAAATTTCG
\end{verbatim}

\newpage


\section{Random Mutation}

\subsection*{Description}
You are given two long sequences of DNAs as two strings.  The two strings are called $A$ and $B$ for convenience, and their length is equal, being $n$.
You are going to mutate them at random as follows:
\begin{itemize}
  \item Pick a number $l$ at random which specifies the length of substrings being mutated - note that $l$ should be between $1$ and $n$
  \item Pick a number $a$ at random, at which the mutation starts in string $A$ - note that $a$ should be between $0$ and $n - l$
  \item Pick a number $b$ at random, at which the mutation starts in string $B$ - note that $a$ should be between $0$ and $n - l$
  \item Finally, you are going to mutate $A$ and $B$ to get a new string $A'$ and $B'$ by swapping the substring $A[a .. (a+l-1)]$ and $B[b .. (b+l-1)]$
\end{itemize}

\subsection*{Note}
We are using a file input (use FILE, fopen, fscanf) and the standard console output (printf).

\subsection*{Input Format}
You are given two strings whose length is equal and at least 20 bytes long, consisting only of A, C, G, and T.

\subsection*{Sample Input (input.txt)}
\begin{verbatim}
ACACAGGGTTAAAAATTTCG
TAGACAGTACTGACTAATGC
\end{verbatim}

\subsection*{Sample Output}
\begin{verbatim}
A: ACACAGGGTTAAAAATTTCG
B: TAGACAGTACTGACTAATGC
Random numbers: l = 4, a = 0, b = 10
    01234567890123456789
A': TGACAGGGTTAAAAATTTCG
B': TAGACAGTACACACTAATGC
\end{verbatim}



\end{document}



