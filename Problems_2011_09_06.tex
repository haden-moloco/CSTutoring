\documentclass[12pt]{article}
%\usepackage{fullpage}
\usepackage{epic}
\usepackage{eepic}
\usepackage{amsmath,amsthm,amsfonts,amssymb}
\usepackage{mathrsfs}
\usepackage{tweaklist}
\usepackage{algorithmic,algorithm}
\usepackage{ulem}

%%%%%%%%%%%%%%%%%%%%%%%%%%%%%%%%%%%%%%%%%%%%%%%%%%%%%%%%%%%%%%%%
% This is FULLPAGE.STY by H.Partl, Version 2 as of 15 Dec 1988.
% Document Style Option to fill the paper just like Plain TeX.

\typeout{Style Option FULLPAGE Version 2 as of 15 Dec 1988}

\topmargin 0pt
\advance \topmargin by -\headheight
\advance \topmargin by -\headsep

\textheight 8.9in

\oddsidemargin 0pt
\evensidemargin \oddsidemargin
\marginparwidth 0.5in

\textwidth 6.5in
%%%%%%%%%%%%%%%%%%%%%%%%%%%%%%%%%%%%%%%%%%%%%%%%%%%%%%%%%%%%%%%%

\pagestyle{empty}
\setlength{\oddsidemargin}{0in}
\setlength{\topmargin}{-0.8in}
\setlength{\textwidth}{6.8in}
\setlength{\textheight}{9.5in}
\setlength{\parindent}{0in}
\setlength{\parskip}{1ex}

\input{code}

\newtheorem{prop}{Proposition}
\newtheorem{lemma}{Lemma}
\newtheorem{thm}{Theorem}

\newcommand{\cost}{\operatorname{cost}}
\newcommand{\opt}{\mathrm{OPT}}

\usepackage{graphicx}
\begin{document}

\setlength{\fboxrule}{.5mm}\setlength{\fboxsep}{1.2mm}
\newlength{\boxlength}\setlength{\boxlength}{\textwidth}
\addtolength{\boxlength}{-4mm}

\def\ind{\hspace*{0.3in}}
\def\gap{0.2in}



\begin{center}\framebox{\parbox{\boxlength}{\bf
CS Tutoring \hfill Author: Haden Lee\\ 
Problem Set \#6 \hfill Date: 09/06/2011}}\end{center}
\vspace{5mm}


\section{Binary Strings}

\subsection*{Description}
Given an integer $n$, output all binary strings of length $n$ in lexicographic order. See sample input/output for details.

\subsection*{Note}
You MUST use a recursive function. Use the following signature:
\begin{verbatim}

char ans[128];

// index goes from 0 to n-1
void factorial(int index) {
  // your implementation goes here
  // (1) what is the terminating condition? upon which, you output ans[]
  // (2) how do you construct the next string?
}
\end{verbatim}

\subsection*{Input Format}
A single integer $n$ is given. $1\leq n \leq 6$. 

\subsection*{Output Format}
Output all binary strings whose length is $n$. 

\subsection*{Sample Input}
\begin{verbatim}
3
\end{verbatim}

\subsection*{Sample Output}
\begin{verbatim}
000
001
010
011
100
101
110
111
\end{verbatim}


\newpage

\section{Permutations}

\subsection*{Description}
Given $n$, output all permutations of integers $1, 2, \dots, n$ (there are total $n!$ of them) in lexicographic order. 

\subsection*{Note}
You MUST use a recursive function. Use the following signature:
\begin{verbatim}
int ans[128];
int used[128];

void perm(int index) {
  // your implementation goes here
  // (1) what is the terminating condition? upon which, you output the numbers in ans[]
  // (2) use 'used[]' array to check whether some number has already been used or not
  return 0;
}
\end{verbatim}

\subsection*{Input Format}
A single integer $n$ is given. $1 \leq n \leq 5$. 

\subsection*{Output Format}
Output all permutations.

\subsection*{Sample Input}
\begin{verbatim}
3
\end{verbatim}

\subsection*{Sample Output}
\begin{verbatim}
1 2 3
1 3 2
2 1 3
2 3 1
3 1 2
3 2 1
\end{verbatim}


\newpage


\section{A+B+C}

\subsection*{Description}
Given three integers, output the sum of the three. 

\subsection*{Input Format}
Three integers are given.  Each integer ranges in $1 \dots 10^{60}$. 

\subsection*{Output Format}
Output the sum.

\subsection*{Sample Input}
\begin{verbatim}
123123123123123123123123123123123123
123123123123123123123123123123123123
123123123123123123123123123123123123
\end{verbatim}

\subsection*{Sample Output}
\begin{verbatim}
369369369369369369369369369369369369
\end{verbatim}

\newpage



\section{Anagram}

\subsection*{Description}
An anagram of a word is defined as another word (possibly the same word) that could be obtained by re-arranging the alphabets in the original word.

For instance, abc is an anagram of cba or bac. Yet abc is not an anagram of aabc because it is missing an 'a'.

Anagram is further defined to a set of words. For instance, George Bush is an anagram of He bugs Gore (not case-sensitive, in this case). 

Your job is, given two words, to check whether the two words are anagrams of each other. 


\subsection*{Input Format}
Two strings are given.  Each string contains lower-case alphabets only.

\subsection*{Output Format}
Output 'Anagram' or 'Not Anagram'. 

\subsection*{Sample Input}
\begin{verbatim}
abc cba
\end{verbatim}

\subsection*{Sample Output}
\begin{verbatim}
Anagram
\end{verbatim}

\subsection*{Sample Input 2}
\begin{verbatim}
asdf xyz
\end{verbatim}

\subsection*{Sample Output 2}
\begin{verbatim}
Not Anagram
\end{verbatim}

\newpage


\input{Footer}
