\documentclass[12pt]{article}
%\usepackage{fullpage}
\usepackage{epic}
\usepackage{eepic}
\usepackage{amsmath,amsthm,amsfonts,amssymb}
\usepackage{mathrsfs}
\usepackage{tweaklist}
\usepackage{algorithmic,algorithm}
\usepackage{ulem}

%%%%%%%%%%%%%%%%%%%%%%%%%%%%%%%%%%%%%%%%%%%%%%%%%%%%%%%%%%%%%%%%
% This is FULLPAGE.STY by H.Partl, Version 2 as of 15 Dec 1988.
% Document Style Option to fill the paper just like Plain TeX.

\typeout{Style Option FULLPAGE Version 2 as of 15 Dec 1988}

\topmargin 0pt
\advance \topmargin by -\headheight
\advance \topmargin by -\headsep

\textheight 8.9in

\oddsidemargin 0pt
\evensidemargin \oddsidemargin
\marginparwidth 0.5in

\textwidth 6.5in
%%%%%%%%%%%%%%%%%%%%%%%%%%%%%%%%%%%%%%%%%%%%%%%%%%%%%%%%%%%%%%%%

\pagestyle{empty}
\setlength{\oddsidemargin}{0in}
\setlength{\topmargin}{-0.8in}
\setlength{\textwidth}{6.8in}
\setlength{\textheight}{9.5in}
\setlength{\parindent}{0in}
\setlength{\parskip}{1ex}

% -*- Tex -*- macros for no-fill, built on Plain

\newenvironment{code}{\samepage \vskip 0.1in \tt\nofill}{\endnofill \vskip 0.1in}
\newenvironment{tightcode}{\samepage \tt\nofill}{\endnofill}

\catcode`\@=11\relax	     % allow @ in macro names

% a simple no-fill, used as
%
% \nofill
% Here is some no-fill text.
%    This line starts with three spaces.
% This is the    last line.  It has embedded spaces which will appear.
% \endnofill
%
% You might want to switch to a fixed-width font when you use
% these macros.

\def\nofill {%
  \begingroup
    % spaces made active so they can be tested for at start of lines
    %  and so multiple spaces aren't collapsed to one.
    \obeyspaces
    % nofill via every line a par; if spaces start next line
    %   we do a \noindent to make sure a par gets started.
    \parskip=\z@
    \parindent=\z@
    \let\p@r=\par
    \def\par{\p@r \ifspacenext{\noindent}{}}%
    \obeylines}

\def\endnofill{
  \endgroup}

% \ifspacenext is used as:
%     \ifspacenext {<true tokens>}{<false tokens>}
% It executes the true tokens if
% the next character is an active space, the false tokens
% otherwise.  The same technique can be used to check for any
% character, not just active space.

% You can NOT nest uses of \ifspacenext!

% get a token which is \ifx-equal to active space, so we can test
% for it
{\obeyspaces\global\let\sp@ce= \relax}

\def\ifspacenext #1#2{%
  \def\truet@ks{#1}%
  \def\falset@ks{#2}%
  \futurelet\next\ifsp@cenext}
\def\ifsp@cenext {%
  \ifx\next\sp@ce \truet@ks \else \falset@ks \fi}

\catcode`\@=12\relax	% make @ inaccessible again


\newtheorem{prop}{Proposition}
\newtheorem{lemma}{Lemma}
\newtheorem{thm}{Theorem}

\newcommand{\cost}{\operatorname{cost}}
\newcommand{\opt}{\mathrm{OPT}}

\usepackage{graphicx}
\begin{document}

\setlength{\fboxrule}{.5mm}\setlength{\fboxsep}{1.2mm}
\newlength{\boxlength}\setlength{\boxlength}{\textwidth}
\addtolength{\boxlength}{-4mm}

\def\ind{\hspace*{0.3in}}
\def\gap{0.2in}



\begin{center}\framebox{\parbox{\boxlength}{\bf
CS Tutoring \hfill Author: Haden Lee\\ 
Problem Set \#3 \hfill Date: 08/25/2011}}\end{center}
\vspace{5mm}


\section{Palindrome 1}

\subsection*{Description}
A palindrome is a string that reads the same in either direction.
For instance, 'dog', 'apple', and 'computer' are not palindromes, but 'madam', 'aka', and 'noon' are.

Your job is to check whether a given string is a palindrome.

\subsection*{Input Format}
You are given a string that only consists of lower-case letters, whose length is between $1$ and $100$. 

\subsection*{Output Format}
Output "Palindrome" if the given string is a palindrome.
Otherwise, output "Not Palindrome".


\subsection*{Sample Input 1}
\begin{verbatim}
dog
\end{verbatim}

\subsection*{Sample Output 1}
\begin{verbatim}
Not Palindrome
\end{verbatim}

\subsection*{Sample Input 2}
\begin{verbatim}
noon
\end{verbatim}

\subsection*{Sample Output 2}
\begin{verbatim}
Palindrome
\end{verbatim}


\newpage

\section{Lottery Game 1}

\subsection*{Description}
Consider the following simple lottery game: You are to choose $m$ numbers between $1$ and $n$ (where $1 \leq m \leq n$). At the end of the day, the seller will also choose $m$ numbers like you did, and if you get $k$ or more numbers correct, you win. What is the probability of winning?

\subsection*{Input Format}
You are given three integers: $n$, $m$, and $k$. 
It is always the case that $1 \leq k \leq m\leq n \leq 10$. 

\subsection*{Output Format}
Output your winning chance, rounded up to the $6$th decimal point.
\textbf{Note} Use type double (instead of float), and use \%.6lf in your printf to print up to the 6th decimal point (it rounds up automatically).


\subsection*{Sample Input 1}
\begin{verbatim}
3 2 1
\end{verbatim}

\subsection*{Sample Output 1}
\begin{verbatim}
1.000000
\end{verbatim}

\subsection*{Sample Input 2}
\begin{verbatim}
3 1 1
\end{verbatim}

\subsection*{Sample Output 2}
\begin{verbatim}
0.333333
\end{verbatim}


\subsection*{Sample Input 3}
\begin{verbatim}
8 2 1
\end{verbatim}

\subsection*{Sample Output 3}
\begin{verbatim}
0.464286
\end{verbatim}

\newpage


\section{Selection Sort}

\subsection*{Description}
\begin{verbatim}
http://www.algolist.net/Algorithms/Sorting/Selection_sort
\end{verbatim}

Read the tutorial and write YOUR OWN C++ CODE that sorts the given numbers.

In short, selection sort works as follows (for $n$ numbers):

(Step 0) You scan through the list from index $0$ to $n-1$, find the smallest number. Swap that smallest number with the number at index $0$ (hence, moving it to the front of the list).

(Step 1) You scan through the list from index $1$ to $n-1$, find the smallest number. Swap that smallest number with the number at index $1$ (hence, moving it to the front of the list).

(Step 2) You scan through the list from index $2$ to $n-1$, find the smallest number. Swap that smallest number with the number at index $2$ (hence, moving it to the front of the list). And so on.

\subsection*{Input Format}
You are given $n$, which is the number of integers that will be given. Assume $1 \leq n \leq 100$.
Next, you will be given $n$ integers, ranged between $-100$ and $100$. They are not necessarily distinct. 

\subsection*{Output Format}
Output the sorted list of $n$ numbers.

\subsection*{Sample Input 1}
\begin{verbatim}
3
2 1 3
\end{verbatim}

\subsection*{Sample Output 1}
\begin{verbatim}
1 2 3
\end{verbatim}

\subsection*{Sample Input 2}
\begin{verbatim}
7
10 20 30 40 -2 -4 -5 
\end{verbatim}

\subsection*{Sample Output 2}
\begin{verbatim}
-5 -4 -2 10 20 30 40
\end{verbatim}




\end{document}

