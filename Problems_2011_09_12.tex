\documentclass[12pt]{article}
%\usepackage{fullpage}
\usepackage{epic}
\usepackage{eepic}
\usepackage{amsmath,amsthm,amsfonts,amssymb}
\usepackage{mathrsfs}
\usepackage{tweaklist}
\usepackage{algorithmic,algorithm}
\usepackage{ulem}

%%%%%%%%%%%%%%%%%%%%%%%%%%%%%%%%%%%%%%%%%%%%%%%%%%%%%%%%%%%%%%%%
% This is FULLPAGE.STY by H.Partl, Version 2 as of 15 Dec 1988.
% Document Style Option to fill the paper just like Plain TeX.

\typeout{Style Option FULLPAGE Version 2 as of 15 Dec 1988}

\topmargin 0pt
\advance \topmargin by -\headheight
\advance \topmargin by -\headsep

\textheight 8.9in

\oddsidemargin 0pt
\evensidemargin \oddsidemargin
\marginparwidth 0.5in

\textwidth 6.5in
%%%%%%%%%%%%%%%%%%%%%%%%%%%%%%%%%%%%%%%%%%%%%%%%%%%%%%%%%%%%%%%%

\pagestyle{empty}
\setlength{\oddsidemargin}{0in}
\setlength{\topmargin}{-0.8in}
\setlength{\textwidth}{6.8in}
\setlength{\textheight}{9.5in}
\setlength{\parindent}{0in}
\setlength{\parskip}{1ex}

\input{code}

\newtheorem{prop}{Proposition}
\newtheorem{lemma}{Lemma}
\newtheorem{thm}{Theorem}

\newcommand{\cost}{\operatorname{cost}}
\newcommand{\opt}{\mathrm{OPT}}

\usepackage{graphicx}
\begin{document}

\setlength{\fboxrule}{.5mm}\setlength{\fboxsep}{1.2mm}
\newlength{\boxlength}\setlength{\boxlength}{\textwidth}
\addtolength{\boxlength}{-4mm}

\def\ind{\hspace*{0.3in}}
\def\gap{0.2in}



\begin{center}\framebox{\parbox{\boxlength}{\bf
CS Tutoring \hfill Author: Haden Lee\\ 
Problem Set \#7 \hfill Date: 09/12/2011}}\end{center}
\vspace{5mm}


\section{Permutations 2}

\subsection*{Description}
Given an integer $n$, output all possible strings that consist only of the first $n$ alphabets.  The strings must contain all $n$ alphabets and must be of length $n$.


\subsection*{Note}
You should solve the original Permutation problem before you try to solve this problem.

\subsection*{Input Format}
An integer $n$, between $1$ and $6$.

\subsection*{Output Format}
See sample input/output. Output the strings in lexicographical order.

\subsection*{Sample Input}
\begin{verbatim}
3
\end{verbatim}

\subsection*{Sample Output}
\begin{verbatim}
abc
acb
bac
bca
cab
cba
\end{verbatim}


\newpage

\section{Selection Sort 2}

\subsection*{Description}
You are given $8$ words.  Sort them lexicographically. 

\subsection*{Note}
Use strcmp(,) function (in string or cstring library) to compare two strings.
For instance, run the following code to see what happens:
\begin{verbatim}
char x[12] = "hello";
char y[12] = "abcd";
char z[12] = "xyz";

printf("%d %d\n", strcmp(x, y), strcmp(y, x));
printf("%d %d\n", strcmp(x, z), strcmp(z, x));
printf("%d %d\n", strcmp(z, y), strcmp(y, z));
\end{verbatim}

You can think of strcmp(str1, str2) as [str1 - str2].  That is, if strcmp returns zero, that means str1 and str2 are identical. if it returns a negative number, that means str2 is bigger (and thus it comes later in dictionary). if it returns a positive number, that means str2 is smaller (and thus it comes before str1 in dictionary). 
Hence, you can compare two strings by calling strcmp, and just use the returned numercial value. strcmp returns an integer value. 

\subsection*{Input Format}
You are given $8$ English words. Sort them. The words consist only of lower-case letters. 

\subsection*{Output Format}
As in sample output.

\subsection*{Sample Input}
\begin{verbatim}
alex
flora
grace
bob
david
emma
chris
helen
\end{verbatim}

\subsection*{Sample Output}
\begin{verbatim}
alex
bob
chris
david
emma
flora
grace
helen
\end{verbatim}


\input{Footer}
