\documentclass[12pt]{article}
%\usepackage{fullpage}
\usepackage{epic}
\usepackage{eepic}
\usepackage{amsmath,amsthm,amsfonts,amssymb}
\usepackage{mathrsfs}
\usepackage{tweaklist}
\usepackage{algorithmic,algorithm}
\usepackage{ulem}

%%%%%%%%%%%%%%%%%%%%%%%%%%%%%%%%%%%%%%%%%%%%%%%%%%%%%%%%%%%%%%%%
% This is FULLPAGE.STY by H.Partl, Version 2 as of 15 Dec 1988.
% Document Style Option to fill the paper just like Plain TeX.

\typeout{Style Option FULLPAGE Version 2 as of 15 Dec 1988}

\topmargin 0pt
\advance \topmargin by -\headheight
\advance \topmargin by -\headsep

\textheight 8.9in

\oddsidemargin 0pt
\evensidemargin \oddsidemargin
\marginparwidth 0.5in

\textwidth 6.5in
%%%%%%%%%%%%%%%%%%%%%%%%%%%%%%%%%%%%%%%%%%%%%%%%%%%%%%%%%%%%%%%%

\pagestyle{empty}
\setlength{\oddsidemargin}{0in}
\setlength{\topmargin}{-0.8in}
\setlength{\textwidth}{6.8in}
\setlength{\textheight}{9.5in}
\setlength{\parindent}{0in}
\setlength{\parskip}{1ex}

% -*- Tex -*- macros for no-fill, built on Plain

\newenvironment{code}{\samepage \vskip 0.1in \tt\nofill}{\endnofill \vskip 0.1in}
\newenvironment{tightcode}{\samepage \tt\nofill}{\endnofill}

\catcode`\@=11\relax	     % allow @ in macro names

% a simple no-fill, used as
%
% \nofill
% Here is some no-fill text.
%    This line starts with three spaces.
% This is the    last line.  It has embedded spaces which will appear.
% \endnofill
%
% You might want to switch to a fixed-width font when you use
% these macros.

\def\nofill {%
  \begingroup
    % spaces made active so they can be tested for at start of lines
    %  and so multiple spaces aren't collapsed to one.
    \obeyspaces
    % nofill via every line a par; if spaces start next line
    %   we do a \noindent to make sure a par gets started.
    \parskip=\z@
    \parindent=\z@
    \let\p@r=\par
    \def\par{\p@r \ifspacenext{\noindent}{}}%
    \obeylines}

\def\endnofill{
  \endgroup}

% \ifspacenext is used as:
%     \ifspacenext {<true tokens>}{<false tokens>}
% It executes the true tokens if
% the next character is an active space, the false tokens
% otherwise.  The same technique can be used to check for any
% character, not just active space.

% You can NOT nest uses of \ifspacenext!

% get a token which is \ifx-equal to active space, so we can test
% for it
{\obeyspaces\global\let\sp@ce= \relax}

\def\ifspacenext #1#2{%
  \def\truet@ks{#1}%
  \def\falset@ks{#2}%
  \futurelet\next\ifsp@cenext}
\def\ifsp@cenext {%
  \ifx\next\sp@ce \truet@ks \else \falset@ks \fi}

\catcode`\@=12\relax	% make @ inaccessible again


\newtheorem{prop}{Proposition}
\newtheorem{lemma}{Lemma}
\newtheorem{thm}{Theorem}

\newcommand{\cost}{\operatorname{cost}}
\newcommand{\opt}{\mathrm{OPT}}

\usepackage{graphicx}
\begin{document}

\setlength{\fboxrule}{.5mm}\setlength{\fboxsep}{1.2mm}
\newlength{\boxlength}\setlength{\boxlength}{\textwidth}
\addtolength{\boxlength}{-4mm}

\def\ind{\hspace*{0.3in}}
\def\gap{0.2in}



\begin{center}\framebox{\parbox{\boxlength}{\bf
CS Tutoring \hfill Author: Haden Lee\\ 
Problem Set \#1 \hfill Date: 08/18/2011}}\end{center}
\vspace{5mm}


\section{Two Missing Numbers}
\subsection*{Description}
\textit{This problem is similar to the missing number problem you saw before.}

You are given a natural number $N$ where $3 \leq N \leq 100$. 
Then you are also given $N$ distinct natural numbers ranged between $1$ and $N+2$, inclusive.  You want to find the two missing numbers. 

\subsection*{Input Format}
You are first given $N$, and then you are given $N$ numbers as described above. 

\subsection*{Output Format}
You output the two missing numbers in order. 

\subsection*{Sample Input 1}
\begin{verbatim}
5
5 7 1 4 2
\end{verbatim}
\subsection*{Sample Output 1}
\begin{verbatim}
3 6
\end{verbatim}

\subsection*{Sample Input 2}
\begin{verbatim}
8
1 2 3 4 7 8 9 10
\end{verbatim}
\subsection*{Sample Output 2}
\begin{verbatim}
5 6
\end{verbatim}



\newpage

\section{Caesar Cipher}
\subsection*{Description}
\textit{From Wikipedia} 

In cryptography, a Caesar cipher is one of the simplest known encryption techniques.  It is a type of substitution cipher in which each letter in the plain-text is replaced by a letter \textit{some fixed number of positions} down the alphabet. For example, with a shift of $3$, A would be replaced by D, B would become E, and so on. The method is named after Julius Caesar, who used it to communicate with his generals.


The transformation can be represented by aligning two alphabets; the cipher alphabet is the plain alphabet rotated left or right by some number of positions. For instance, here is a Caesar cipher using a left rotation of three places (the shift parameter, here 3, is used as the key):
\begin{verbatim}
Plain:    ABCDEFGHIJKLMNOPQRSTUVWXYZ
Cipher:   DEFGHIJKLMNOPQRSTUVWXYZABC
\end{verbatim}

When encrypting, a person looks up each letter of the message in the "plain" line and writes down the corresponding letter in the "cipher" line. Deciphering is done in reverse.
\begin{verbatim}
Ciphertext: WKH TXLFN EURZQ IRA MXPSV RYHU WKH ODCB GRJ
Plaintext:  the quick brown fox jumps over the lazy dog
\end{verbatim}

\subsection*{Input Format}
You are given a string which may be a cipher-text or plain-text. The given string consists only of upper-case alphabets.  Then, you are given an integer that denotes the shift that was used in encryption or should be used in decryption.  This integer ranges between $1$ and $25$, inclusive. 

\subsection*{Output Format}
You first output the encrypted string using the given shift.
Next, you output the decrypted string using the given shift. 

\subsection*{Sample Input 1}
\begin{verbatim}
GRJ
3
\end{verbatim}
\subsection*{Sample Output 1}
\begin{verbatim}
JUM
DOG
\end{verbatim}

\subsection*{Sample Input 2}
\begin{verbatim}
ABCDE
1
\end{verbatim}
\subsection*{Sample Output 2}
\begin{verbatim}
BCDEF
ZABCD
\end{verbatim}


\newpage

\section{Fibonacci Numbers}
\subsection*{Description}
In Mathematics, the Fibonacci numbers are defined as follows:

\begin{equation*}
F_n = F_{n-1} + F_{n-2}
\end{equation*}

with seed values $F_0 = 0$ and $F_1 = 1$. 

Given some number $n$, you want to compute the value of $F_n$. 

\subsection*{Input Format}
You are given an integer $n$ where $0 \leq n \leq 1,000$. 

\subsection*{Output Format}
Output the value of $F_n$.  Since $F_n$ could become really large, you should output $F_n$ modulo 100,007 (that is the remainder of $F_n$ divided by 100,007).

\subsection*{Sample Input 1}
\begin{verbatim}
5
\end{verbatim}
\subsection*{Sample Output 1}
\begin{verbatim}
5
\end{verbatim}

\subsection*{Sample Input 2}
\begin{verbatim}
12
\end{verbatim}
\subsection*{Sample Output 2}
\begin{verbatim}
144
\end{verbatim}






\newpage

\section{Flower Shop}
\subsection*{Description}
You are the owner of a really big flower shop that delivers flowers to customers.  On a particular day, you decided to have the following promotional event.  

You ask $N$ customers to write down the highest price they are willing to pay, provided that you deliver the most beautiful flower bouquet in time.  After collecting the $N$ numbers (the prices), you will decide the border line price $p$ such that any customer who is willing to pay $p$ or higher will pay $p$ dollars and receive the bouquet, but all other customers who are willing to pay strictly less than $p$ will NOT pay and will NOT receive any bouquets. 

As a businessman, you want to maximize profit by choosing the optimal value $p$.  You want to write a program that finds such $p$ for you. 

\subsection*{Input Format}
You are first given the number of customers, $N$ where $2 \leq N \leq 20$. Then, you are given $N$ natural numbers ($m_1, m_2, \dots, m_N$) where $m_i$ represents the maximum amount of money that customer $i$ is willing to pay where $1 \leq i \leq N$. You can assume that $1 \leq m_i \leq 100$ for all $m_i$ where $1 \leq i \leq N$. 

\subsection*{Output Format}
You first output the optimal price $p$, and then output your maximized profit. If there are multiple $p$ values that maximize your profit, you may output any such $p$. 

\subsection*{Sample Input 1}
\begin{verbatim}
5
2 3 2 3 3
\end{verbatim}
\subsection*{Sample Output 1}
\begin{verbatim}
2 10
\end{verbatim}

\subsection*{Note 1}
There is no point for you to set $p > 3$ as your profit will be zero. If you set $2 < p \leq 3$, then you can only collect money from customer $2, 4,$ and $5$, which makes your profit equal to $3p$ dollars; obviously, you will set $p = 3$ in this case to make $9$ dollars. 
If you set $p = 2$, you will collect $2$ dollars from all five customers to obtain $10$ dollars of profit (the optimal solution).
If you set $p < 2$, you will collect $p$ dollars from all customers, but it will always be less than $10$ dollars. 
Hence, you print $2 10$ to denote that you chose the price to be $2$ and your maximized profit was $10$. 

\subsection*{Sample Input 2}
\begin{verbatim}
5
100 100 10 10 10
\end{verbatim}
\subsection*{Sample Output 2}
\begin{verbatim}
100 200
\end{verbatim}



\end{document}

