\documentclass[12pt]{article}
%\usepackage{fullpage}
\usepackage{epic}
\usepackage{eepic}
\usepackage{amsmath,amsthm,amsfonts,amssymb}
\usepackage{mathrsfs}
\usepackage{tweaklist}
\usepackage{algorithmic,algorithm}
\usepackage{ulem}

%%%%%%%%%%%%%%%%%%%%%%%%%%%%%%%%%%%%%%%%%%%%%%%%%%%%%%%%%%%%%%%%
% This is FULLPAGE.STY by H.Partl, Version 2 as of 15 Dec 1988.
% Document Style Option to fill the paper just like Plain TeX.

\typeout{Style Option FULLPAGE Version 2 as of 15 Dec 1988}

\topmargin 0pt
\advance \topmargin by -\headheight
\advance \topmargin by -\headsep

\textheight 8.9in

\oddsidemargin 0pt
\evensidemargin \oddsidemargin
\marginparwidth 0.5in

\textwidth 6.5in
%%%%%%%%%%%%%%%%%%%%%%%%%%%%%%%%%%%%%%%%%%%%%%%%%%%%%%%%%%%%%%%%

\pagestyle{empty}
\setlength{\oddsidemargin}{0in}
\setlength{\topmargin}{-0.8in}
\setlength{\textwidth}{6.8in}
\setlength{\textheight}{9.5in}
\setlength{\parindent}{0in}
\setlength{\parskip}{1ex}

\input{code}

\newtheorem{prop}{Proposition}
\newtheorem{lemma}{Lemma}
\newtheorem{thm}{Theorem}

\newcommand{\cost}{\operatorname{cost}}
\newcommand{\opt}{\mathrm{OPT}}

\usepackage{graphicx}
\begin{document}

\setlength{\fboxrule}{.5mm}\setlength{\fboxsep}{1.2mm}
\newlength{\boxlength}\setlength{\boxlength}{\textwidth}
\addtolength{\boxlength}{-4mm}

\def\ind{\hspace*{0.3in}}
\def\gap{0.2in}



\begin{center}\framebox{\parbox{\boxlength}{\bf
CS Tutoring \hfill Author: Haden Lee\\ 
Problem Set \#5 \hfill Date: 09/04/2011}}\end{center}
\vspace{5mm}


\section{Factorial}

\subsection*{Description}
Compute $n!$ when a positive integer $n$ is given.  Since the number could be very huge, you have to compute $(n!)$ modulo $1,000,007$. 

\subsection*{Note}
You MUST use a recursive function. Use the following signature:
\begin{verbatim}

int factorial(int x) {
  // your implementation goes here
  // (1) define your 'base case'
  // (2) solve a recurrence relation (in this case, it is n! = ((n-1)!) * n
  return 0;
}
\end{verbatim}

\subsection*{Input Format}
A single integer $n$ is given. Assume $n \leq 1,000$.

\subsection*{Output Format}
Output $(n!)$ modulo $1,000,007$. 

\subsection*{Sample Input}
\begin{verbatim}
10
\end{verbatim}

\subsection*{Sample Output}
\begin{verbatim}
362880
\end{verbatim}

\subsection*{Sample Input 2}
\begin{verbatim}
20
\end{verbatim}

\subsection*{Sample Output 2}
\begin{verbatim}
794133
\end{verbatim}



\newpage

\section{Powers}

\subsection*{Description}
Compute $n^p$ when you are given two positive integers $n$ and $p$. Since this number could be huge, compute $n^p$ modulo $1,000,007$. 

\subsection*{Note}
You MUST use a recursive function. Use the following signature:
\begin{verbatim}

int power(int n, int p) {
  // your implementation goes here
  // (1) define your 'base case'
  // (2) solve a recurrence relation (in this case, it is n^p = (n^{p-1}) * n
  return 0;
}
\end{verbatim}

\subsection*{Input Format}
Two integers $n$ and $p$ are given. Assume $1\leq p \leq 100$ and $1 \leq n \leq 100$.

\subsection*{Output Format}
Output $n^p$ modulo $1,000,007$. 

\subsection*{Sample Input}
\begin{verbatim}
5 2
\end{verbatim}

\subsection*{Sample Output}
\begin{verbatim}
25
\end{verbatim}

\subsection*{Sample Input 2}
\begin{verbatim}
2 20
\end{verbatim}

\subsection*{Sample Output 2}
\begin{verbatim}
48569
\end{verbatim}

\newpage


\section{Fibonacci, Take Two}

\subsection*{Description}
Compute the $n$-th Fibonacci number, given $n$. 
$F_0 = F_1 = 1$.  Since $F_n$ could be huge, compute $F_n$ modulo $1,000,007$. 

\subsection*{Note}
You MUST use a recursive function. Use the following signature:
\begin{verbatim}

int fibo(int n) {
  // your implementation goes here
  // (1) define your 'base case'
  // (2) solve a recurrence relation (in this case, it is F_{n} = F_{n-1} + F_{n-2})
  return 0;
}
\end{verbatim}

\subsection*{Note2}
Since your function fibo() will be called exponentially many times, declare a global variable int answer[], and check if the answer was ALREADY computed previously by comb(). If so, you can just return the value in ans[n]; otherwise, you compute the value AND STORE it in ans[n].

\subsection*{Input Format}
An integer $n$ is given. $n$ is always between $0$ and $1,000$. 

\subsection*{Output Format}
Output $F_n$ modulo $1,000,007$. 

\subsection*{Sample Input}
\begin{verbatim}
0
\end{verbatim}

\subsection*{Sample Output}
\begin{verbatim}
1
\end{verbatim}

\subsection*{Sample Input 2}
\begin{verbatim}
5
\end{verbatim}

\subsection*{Sample Output 2}
\begin{verbatim}
5
\end{verbatim}

\newpage



\section{Pascal, Take Two}

\subsection*{Description}
Compute $n$ choose $r$ (formally, $\binom{n}{r} = \frac{n!}{r! \cdot (n-r)!}$).  Since the number could be very huge, you have to compute $\binom{n}{r}$ modulo $1,000,007$. 

\subsection*{Note}
You MUST use a recursive function. Use the following signature:
\begin{verbatim}

int comb(int n, int r) {
  // your implementation goes here
  // (1) define your 'base case'
  // (2) solve a recurrence relation (in this case, it is n choose r = (n-1 choose r-1) + (n choose r-1)
  return 0;
}
\end{verbatim}

\subsection*{Note2}
Since your function comb() will be called exponentially many times, declare a global variable int answer[][], and check if the answer was ALREADY computed previously by comb(). If so, you can just return the value in ans[n][r]; otherwise, you compute the value AND STORE it in ans[n][r].

\subsection*{Input Format}
Two integers $n$ and $r$ are given. Assume $0\leq r \leq n \leq 1,000$.

\subsection*{Output Format}
Output $\binom{n}{r}$ modulo $1,000,007$. 

\subsection*{Sample Input}
\begin{verbatim}
5 2
\end{verbatim}

\subsection*{Sample Output}
\begin{verbatim}
10
\end{verbatim}

\subsection*{Sample Input 2}
\begin{verbatim}
30 15
\end{verbatim}

\subsection*{Sample Output 2}
\begin{verbatim}
116435
\end{verbatim}

\newpage

\section{Car Plate}

\subsection*{Description}
Given a string, determine whether it is a valid Californian car plate number, according to the following rule:

For simplicity, a valid Californian car plate number:
\begin{itemize}
  \item Must be 7 characters long such that 
  \item the first character is a number between $1$ and $9$,
  \item the following three characters are upper-case letters,
  \item and the following three characters are digits including zeros
\end{itemize}

Valid car plate numbers are: 1ABC123, 9XZY009, and 6SLJ356.
Some invalid car plate numbers are: VF, 0XTZ123, 3aBX334, ASDF999, 8HELLO5, etc.


\subsection*{Input Format}
A string is given, whose length is between $1$ and $10$. 

\subsection*{Output Format}
Output ''Valid'' or ''Invalid'' based on the rule above.

\subsection*{Sample Input}
\begin{verbatim}
1ABC123
\end{verbatim}

\subsection*{Sample Output}
\begin{verbatim}
Valid
\end{verbatim}

\subsection*{Sample Input 2}
\begin{verbatim}
3AbX312
\end{verbatim}

\subsection*{Sample Output 2}
\begin{verbatim}
Invalid
\end{verbatim}

\subsection*{Sample Input 3}
\begin{verbatim}
0XYZ992
\end{verbatim}

\subsection*{Sample Output 3}
\begin{verbatim}
Invalid
\end{verbatim}

\newpage



\input{Footer}
